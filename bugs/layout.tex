%\documentclass[10pt,a4paper,twoside]{article}
\documentclass[11pt,oneside,bibliography=totoc]{scrartcl}
\usepackage[inner=1.3 cm,outer=1.3 cm,top=2.5cm,bottom=2.0cm]{geometry}
\usepackage[utf8]{inputenc}
\usepackage{amsmath}
\usepackage{amsfonts}
\usepackage{amssymb}
\usepackage{hyperref} % makes hyperlinks possible [seems to hyperlink toc automatically :)]
\usepackage[usenames,dvipsnames]{xcolor} % loads colors
\usepackage{fancyhdr} % I think this is for the heading line
\hypersetup{
    colorlinks,
    citecolor=black,
    filecolor=black,
    linkcolor=black,
    urlcolor=black
}
\usepackage{float} % For float boxes
\makeatletter % allows use of "@" in control sequence names
\usepackage{nameref} % Used to link sectionname to \currentname

\begin{document}
\pagestyle{fancy}
\fancyhf{}\
% Set the headline to be a hyperlink to the table of contents
\fancyhead[R]{ \hyperlink{toc}{Table of contents} }




% >>>>>>>>>>>>>>>>>>>>>>>>>>>>>> INHALTSVERZEICHNIS
\hypertarget{toc}{\tableofcontents} % table of contents to be accessed by heading line.

\newpage
% >>>>>>>>>>>>>>>>>>>>>>>>>>>>>> NOMENKLATUR
\section*{Nomenclatur}

% Define new command to indicate special input.

% TERMINAL
\newcommand\terminal{%%
  \bgroup
    \catcode`\%=12 % \catcode`\<char>=<number>
    \catcode`\#=12 % With catcode the letters can be assigned to different groups (0-15)
    \ae@terminal
  }
\def\ae@terminal#1{
\color{NavyBlue}
    \texttt{\detokenize{#1}}
\egroup}

% CPP
\newcommand\cpp{%%
  \bgroup
    \catcode`\%=12 % \catcode`\<char>=<number>
    \catcode`\#=12 % With catcode the letters can be assigned to different groups (0-15)
    \af@cpp
  }
\def\af@cpp#1{
\color{ForestGreen}
    \texttt{\detokenize{#1}}
\egroup}

% Python
\newcommand\python{%%
  \bgroup
    \catcode`\%=12 % \catcode`\<char>=<number>
    \catcode`\#=12 % With catcode the letters can be assigned to different groups (0-15)
    \ag@python
  }
\def\ag@python#1{
\color{Red}
    \texttt{\detokenize{#1}}
\egroup}

% GDB
\newcommand\gdb{%%
  \bgroup
    \catcode`\%=12 % \catcode`\<char>=<number>
    \catcode`\#=12 % With catcode the letters can be assigned to different groups (0-15)
    \ah@gdb
  }
\def\ah@gdb#1{
\color{Plum}
    \texttt{\detokenize{#1}}
\egroup}

% LATEX
\newcommand\latex{%%
  \bgroup
    \catcode`\%=12 % \catcode`\<char>=<number>
    \catcode`\#=12 % With catcode the letters can be assigned to different groups (0-15)
    \ai@latex
  }
\def\ai@latex#1{
\color{CornflowerBlue}
    \texttt{\detokenize{#1}}
\egroup}

% LINK
\newcommand\link{%%
  \bgroup
    \catcode`\%=12 % \catcode`\<char>=<number>
    \catcode`\#=12 % With catcode the letters can be assigned to different groups (0-15)
    \aj@link
  }
\def\aj@link#1{
\color{BurntOrange}
    \texttt{\detokenize{#1}}
\egroup}

% XML
\newcommand\xml{%%
  \bgroup
    \catcode`\%=12 % \catcode`\<char>=<number>
    \catcode`\#=12 % With catcode the letters can be assigned to different groups (0-15)
    \ak@xml
  }
\def\ak@xml#1{
\color{Brown}
    \texttt{\detokenize{#1}}
\egroup}


% Examples
\terminal{This is how a terminal command looks like}\\
\cpp{This is how a C++ command looks like}\\
\python{This is how a Python command looks like}\\
\gdb{This is how a gdb command looks like}\\
\latex{This is how a latex command looks like}\\
\link{This is how a link looks like}\\
\xml{This is how a xml property like}\\

% Tipps:
% Do not use underscores: "missing $ inserted"-error | 
%   if necessary write \section{ \terminal{a\_b} } instead of \section{ \terminal{a_b} }
% 

% Create a 'bug'-Box
% this creates a custom and simpler ruled box style
\newcommand\floatc@simplerule[2]{{\@fs@cfont #1 #2}\par}
\newcommand\fs@simplerule{\def\@fs@cfont{\bfseries}\let\@fs@capt\floatc@simplerule
  \def\@fs@pre{\hrule height.8pt depth0pt \kern4pt}%
  \def\@fs@post{\kern4pt\hrule height.8pt depth0pt \kern4pt \relax}%
  \def\@fs@mid{\kern8pt}%
  \let\@fs@iftopcapt\iftrue}

% this code block defines the new and custom floatbox float environment
\floatstyle{simplerule}
\newfloat{bug}{thp}{lob}
\floatname{bug}{Description of}

% Link section name to \currentname
\makeatletter
\newcommand*{\currentname}{\@currentlabelname}
\makeatother